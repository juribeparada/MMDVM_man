\documentclass[]{article}

\usepackage{hyperref}
\usepackage{listings}
\usepackage{xcolor}
\usepackage{enumitem}

\lstdefinestyle{DOS}
{
	backgroundcolor=\color{black},
	breaklines=true,
	basicstyle=\scriptsize\color{white}\ttfamily
}

\title{Build instructions for MMDVM firmware using STM32F4XX or STM32F7XX}
\author{Andy CA6JAU, Jim KI6ZUM}

\begin{document}

\maketitle
\tableofcontents

\section{Raspberry Pi}

\subsection{Nucleo-64 F446RE}

\begin{itemize}[leftmargin=*]
	
\item If you are using Pi-Star, expand filesystem (if you haven't done before):

\begin{lstlisting}[style=DOS]
sudo pistar-expand
sudo reboot
\end{lstlisting}

\item Enable RW filesystem if you are using Pi-Star:
\begin{lstlisting}[style=DOS]
rpi-rw
\end{lstlisting}

\item Install toolchain and necessary packages:

\begin{lstlisting}[style=DOS]
sudo apt-get install git gcc-arm-none-eabi gdb-arm-none-eabi libstdc++-arm-none-eabi-newlib autoconf libtool pkg-config libusb-1.0-0 libusb-1.0-0-dev
\end{lstlisting}

\item Install OpenOCD:

\begin{lstlisting}[style=DOS]
git clone https://github.com/juribeparada/openocd
cd openocd
./bootstrap
./configure
make
sudo make install
\end{lstlisting}

\item Download the sources:

\begin{lstlisting}[style=DOS]
git clone https://github.com/g4klx/MMDVM
cd MMDVM
git clone https://github.com/juribeparada/STM32F4XX_Lib
\end{lstlisting}

\item Edit Config.h:

\begin{lstlisting}[style=DOS]
nano Config.h
\end{lstlisting}

\item Usually you could enable (for Morpho connector):

\begin{verbatim}
#define ARDUINO_MODE_PINS
#define STM32F4_NUCLEO_MORPHO_HEADER
#define SEND_RSSI_DATA
#define SERIAL_REPEATER
\end{verbatim}

\item Compile the code:

\begin{lstlisting}[style=DOS]
make nucleo
\end{lstlisting}

\item If you are using Pi-Star, stop services:

\begin{lstlisting}[style=DOS]
sudo pistar-watchdog.service stop
sudo systemctl stop mmdvmhost.timer
sudo systemctl stop mmdvmhost.service
\end{lstlisting}

\item Upload the firmware:

\begin{lstlisting}[style=DOS]
sudo make deploy
\end{lstlisting}

\end{itemize}

\subsection{Nucleo-144 F767ZI}

\begin{itemize}[leftmargin=*]

\item If you are using Pi-Star, expand filesystem (if you haven't done before):
\begin{lstlisting}[style=DOS]
sudo pistar-expand
sudo reboot
\end{lstlisting}

\item Install toolchain and necessary packages:
\begin{lstlisting}[style=DOS]
sudo apt-get install git gcc-arm-none-eabi gdb-arm-none-eabi libstdc++-arm-none-eabi-newlib autoconf libtool pkg-config libusb-1.0-0 libusb-1.0-0-dev
\end{lstlisting}

\item Install OpenOCD:
\begin{lstlisting}[style=DOS]
git clone https://github.com/juribeparada/openocd
cd openocd
./bootstrap
./configure
make
sudo make install
\end{lstlisting}

\item Download the sources:
\begin{lstlisting}[style=DOS]
git clone https://github.com/g4klx/MMDVM
cd MMDVM
git clone https://github.com/juribeparada/STM32F7XX_Lib
\end{lstlisting}

\item Edit Config.h according your preferences:
\begin{lstlisting}[style=DOS]
nano Config.h
\end{lstlisting}

\item Compile the code:
\begin{lstlisting}[style=DOS]
make f767
\end{lstlisting}

\item If you are using Pi-Star, stop services:
\begin{lstlisting}[style=DOS]
sudo pistar-watchdog.service stop
sudo systemctl stop mmdvmhost.timer
sudo systemctl stop mmdvmhost.service
\end{lstlisting}

\item Upload the firmware:
\begin{lstlisting}[style=DOS]
sudo make deploy-f7
\end{lstlisting}

\end{itemize}

\subsection{MMDVM-Pi}

\subsubsection{Enable serial port in Raspberry Pi 3 or Pi Zero W}

This this necessary only if you are installing a fresh copy of Raspbian OS. Images like Pi-Star are already OK.

\begin{itemize}[leftmargin=*]

\item Edit /boot/cmdline.txt:
\begin{lstlisting}[style=DOS]
sudo nano /boot/cmdline.txt
\end{lstlisting}
(remove the text: console=serial0,115200)

\item Disable services:
\begin{lstlisting}[style=DOS]
sudo systemctl disable serial-getty@ttyAMA0.service
sudo systemctl disable bluetooth.service
\end{lstlisting}

\item Edit /boot/config.txt:
\begin{lstlisting}[style=DOS]
sudo nano /boot/config.txt
\end{lstlisting}

and add the following lines at the end of /boot/config.txt:
\begin{verbatim}
enable_uart=1
dtoverlay=pi3-disable-bt
\end{verbatim}

\item Reboot your RPi:
\begin{lstlisting}[style=DOS]
sudo reboot
\end{lstlisting}

\end{itemize}

\subsubsection{Installation of necessary software (only once)}

\begin{itemize}[leftmargin=*]
	
\item If you are using Pi-Star, expand filesystem (if you haven't done before):
\begin{lstlisting}[style=DOS]
sudo pistar-expand
sudo reboot
\end{lstlisting}

\item Update package lists:
\begin{lstlisting}[style=DOS]
sudo apt-get update
\end{lstlisting}

\item Enable RW filesystem if you are using Pi-Star:
\begin{lstlisting}[style=DOS]
rpi-rw
\end{lstlisting}

\item Install toolchain and necessary packages:
\begin{lstlisting}[style=DOS]
sudo apt-get install git gcc-arm-none-eabi gdb-arm-none-eabi libstdc++-arm-none-eabi-newlib autoconf libtool pkg-config libusb-1.0-0 libusb-1.0-0-dev
\end{lstlisting}

\item Download and compile serial flashing utilities:
\begin{lstlisting}[style=DOS]
cd ~
git clone https://github.com/jsnyder/stm32ld
cd stm32ld
make
sudo cp stm32ld /usr/local/bin
\end{lstlisting}

\item Remove libi2c-dev and stm32flash packages if you are using Pi-Star:
\begin{lstlisting}[style=DOS]
sudo apt-get remove libi2c-dev
sudo apt-get remove stm32flash
\end{lstlisting}

\item Install the latest stm32flash:
\begin{lstlisting}[style=DOS]
cd ~
git clone https://git.code.sf.net/p/stm32flash/code stm32flash
cd stm32flash
make
sudo make install
\end{lstlisting}

\end{itemize}

\subsubsection{Firmware compilation for MMDVM-Pi}

\begin{itemize}[leftmargin=*]
	
\item Download firmware sources:
\begin{lstlisting}[style=DOS]
cd ~
git clone https://github.com/g4klx/MMDVM
cd MMDVM
git clone https://github.com/juribeparada/STM32F4XX_Lib
\end{lstlisting}

\item Edit Config.h according your preferences:
\begin{lstlisting}[style=DOS]
nano Config.h
\end{lstlisting}

\item You can select for example:

\begin{verbatim}
// #define EXTERNAL_OSC 12000000 (disable any external TCXO)
// #define ARDUINO_DUE_ZUM_V10 (this option doesn't matter for STM32 devices)
#define ARDUINO_MODE_PINS
#define SEND_RSSI_DATA
#define SERIAL_REPEATER
\end{verbatim}

\item Compile:
\begin{lstlisting}[style=DOS]
make pi
\end{lstlisting}

\item If you are using Pi-Star, stop services:

\begin{lstlisting}[style=DOS]
sudo pistar-watchdog.service stop
sudo systemctl stop mmdvmhost.timer
sudo systemctl stop mmdvmhost.service
\end{lstlisting}

\item Upload the firmware:
\begin{lstlisting}[style=DOS]
sudo make deploy-pi
\end{lstlisting}

\end{itemize}

\section{Linux Ubuntu}

\begin{itemize}[leftmargin=*]
	
\item Remove the official package:
\begin{lstlisting}[style=DOS]
sudo apt-get purge binutils-arm-none-eabi gcc-arm-none-eabi gdb-arm-none-eabi libstdc++-arm-none-eabi-newlib libnewlib-arm-none-eabi
\end{lstlisting}

\item Add 3rd party repository:
\begin{lstlisting}[style=DOS]
sudo add-apt-repository ppa:team-gcc-arm-embedded/ppa
sudo apt-get update
\end{lstlisting}

\item Check the GCC package version in the PPA repository:
\begin{lstlisting}[style=DOS]
sudo apt-cache policy gcc-arm-embedded
\end{lstlisting}

\item Install software requirements:
\begin{lstlisting}[style=DOS]
sudo apt-get install build-essential git gcc-arm-embedded qemu-system-arm symlinks expect autoconf libtool pkg-config libusb-1.0-0 libusb-1.0-0-dev
\end{lstlisting}

\item Install latest openOCD tool from the sources:
\begin{lstlisting}[style=DOS]
git clone https://github.com/juribeparada/openocd
cd openocd
./bootstrap
./configure
make
sudo make install
\end{lstlisting}
	
\item Get the latest source code from GitHub:
\begin{lstlisting}[style=DOS]
git clone https://github.com/g4klx/MMDVM
cd MMDVM
git clone https://github.com/juribeparada/STM32F4XX_Lib
git clone https://github.com/juribeparada/STM32F7XX_Lib
\end{lstlisting}
	
\item Clean the directory before build:
\begin{lstlisting}[style=DOS]
make clean
\end{lstlisting}
	
\item Edit Config.h according your preferences.
	
\item Then you can use the following commands to build the source code for the Nucleo-64 F446RE board:
\begin{lstlisting}[style=DOS]
make nucleo
\end{lstlisting}
	
or for the STM32F4 Discovery board:
\begin{lstlisting}[style=DOS]
make dis
\end{lstlisting}
	
or for the MMDVM-Pi board:
\begin{lstlisting}[style=DOS]
make pi
\end{lstlisting}
	
or for the Nucleo-144 F767ZI board:
\begin{lstlisting}[style=DOS]
make f767
\end{lstlisting}
	
\item Upload the firmware using OpenOCD (USB, internal ST-Link), for STM32F4 family:
\begin{lstlisting}[style=DOS]
sudo make deploy
\end{lstlisting}
	
or for STM32F7 family:
\begin{lstlisting}[style=DOS]
sudo make deploy-f7
\end{lstlisting}
	
\end{itemize}

\section{Windows versions $<$ 10}

\begin{itemize}[leftmargin=*]
\item Download and install the ST-Link programming software from here (note: you will have to register with ST to be able to download the installer): \url{http://www.st.com/en/embedded-software/stsw-link004.html}

\item Download and install Git for Windows (use default options):
\url{https://git-scm.com/download/win}

\item Download the GNU ARM embedded toolchain from here:
\url{https://launchpad.net/gcc-arm-embedded/+download}

Install the GNU ARM tools in the default location.

\item Launch the "GCC Command Prompt" from "GNU Tools for ARM Embedded Processors" (Start Menu).

\item Get the latest source code from GitHub (change USERNAME for a valid Windows user name):
\begin{lstlisting}[style=DOS]
cd C:\Users\USERNAME\
git clone https://github.com/g4klx/MMDVM
cd MMDVM
git clone https://github.com/juribeparada/STM32F4XX_Lib
git clone https://github.com/juribeparada/STM32F7XX_Lib
\end{lstlisting}

\item Download the GNU make utility:
\url{http://gnuwin32.sourceforge.net/packages/make.htm}

Download the binaries zip file and extract make.exe and put it in the same directory as Makefile.

Download the dependencies zip file and extract libintl3.dll and libiconv2.dll and put them in the same directory as Makefile.

\item Clean the directory before build:
\begin{lstlisting}[style=DOS]
make clean
\end{lstlisting}

\item Edit Config.h according your preferences.

\item Then you can use the following commands to build the source code for the Nucleo-64 F446RE board:
\begin{lstlisting}[style=DOS]
make nucleo
\end{lstlisting}

or for the STM32F4 Discovery board:
\begin{lstlisting}[style=DOS]
make dis
\end{lstlisting}

or for the MMDVM-Pi board:
\begin{lstlisting}[style=DOS]
make pi
\end{lstlisting}

or for the Nucleo-144 F767ZI board:
\begin{lstlisting}[style=DOS]
make f767
\end{lstlisting}

The .hex file will be in the bin folder.

\end{itemize}

\section{Windows 10 with Ubuntu}

\begin{itemize}[leftmargin=*]
	
\item Download and install the ST-Link programming software from here (note: you will have to register with ST to be able to download the installer): \url{http://www.st.com/en/embedded-software/stsw-link004.html}

\item Install bash using these instructions: \url{http://www.pcworld.com/article/3106463/windows/how-to-get-bash-on-windows-10-with-the-anniversary-update.html}

\item Once you have bash installed, install GCC for ARM. It must be $\geq$ version 4.9:
\begin{lstlisting}[style=DOS]
sudo apt-get install gcc
sudo apt-get install make
sudo apt-get remove gcc-arm-none-eabi
sudo add-apt-repository ppa:team-gcc-arm-embedded/ppa
sudo apt-get install -y git gcc-arm-embedded=5-2015q4-1~trusty1
\end{lstlisting}

\item Make sure git is installed. If not install it with:
\begin{lstlisting}[style=DOS]
sudo apt-get install git
\end{lstlisting}

\item Get the latest source code from GitHub:
\begin{lstlisting}[style=DOS]
git clone https://github.com/g4klx/MMDVM
cd MMDVM
git clone https://github.com/juribeparada/STM32F4XX_Lib
git clone https://github.com/juribeparada/STM32F7XX_Lib
\end{lstlisting}

\item Clean the directory before build:
\begin{lstlisting}[style=DOS]
make clean
\end{lstlisting}

\item Edit Config.h according your preferences.

\item Then you can use the following commands to build the source code for the Nucleo-64 F446RE board:
\begin{lstlisting}[style=DOS]
make nucleo
\end{lstlisting}

or for the STM32F4 Discovery board:
\begin{lstlisting}[style=DOS]
make dis
\end{lstlisting}

or for the MMDVM-Pi board:
\begin{lstlisting}[style=DOS]
make pi
\end{lstlisting}

or for the Nucleo-144 F767ZI board:
\begin{lstlisting}[style=DOS]
make f767
\end{lstlisting}

The .hex file will be in the bin folder.

\end{itemize}

\section{macOS}

\begin{itemize}[leftmargin=*]
	
\item First install Homebrew (not root!):
\begin{lstlisting}[style=DOS]
/usr/bin/ruby -e "$(curl -fsSL https://raw.githubusercontent.com/Homebrew/install/master/install)"
\end{lstlisting}

\item Install additional software:
\begin{lstlisting}[style=DOS]
brew install libusb autogen automake wget pkg-config cmake openocd
\end{lstlisting}

\item Install the ARM GCC toolchain:
\begin{lstlisting}[style=DOS]
wget https://launchpad.net/gcc-arm-embedded/5.0/5-2016-q3-update/+download/gcc-arm-none-eabi-5_4-2016q3-20160926-mac.tar.bz2
tar xvf gcc-arm-none-eabi-5_4-2016q3-20160926-mac.tar.bz2
cd gcc-arm-none-eabi-5_4-2016q3/
sudo cp -Rf * /usr/local/
\end{lstlisting}

\item Get the latest source code from GitHub:
\begin{lstlisting}[style=DOS]
git clone https://github.com/g4klx/MMDVM
cd MMDVM
git clone https://github.com/juribeparada/STM32F4XX_Lib
git clone https://github.com/juribeparada/STM32F7XX_Lib
\end{lstlisting}

\item Clean the directory before build:
\begin{lstlisting}[style=DOS]
make clean
\end{lstlisting}

\item Edit Config.h according your preferences.

\item Then you can use the following commands to build the source code for the Nucleo-64 F446RE board:
\begin{lstlisting}[style=DOS]
make nucleo
\end{lstlisting}

or for the STM32F4 Discovery board:
\begin{lstlisting}[style=DOS]
make dis
\end{lstlisting}

or for the MMDVM-Pi board:
\begin{lstlisting}[style=DOS]
make pi
\end{lstlisting}

or for the Nucleo-144 F767ZI board:
\begin{lstlisting}[style=DOS]
make f767
\end{lstlisting}

\item Upload the firmware using OpenOCD (USB, internal ST-Link), for STM32F4 family:
\begin{lstlisting}[style=DOS]
make deploy
\end{lstlisting}

or for STM32F7 family:
\begin{lstlisting}[style=DOS]
make deploy-f7
\end{lstlisting}

\end{itemize}

\section{Pin definitions for different STM32 boards}

\subsection{Pin definitions for STM32F4 Discovery Board:}

\begin{verbatim}
PTT      PB13   output           P1 Pin37
COSLED   PA7    output           P1 Pin17
LED      PD15   output           P1 Pin47
COS      PA5    input            P1 Pin15

DSTAR    PD12   output           P1 Pin44
DMR      PD13   output           P1 Pin45
YSF      PD14   output           P1 Pin46
P25      PD11   output           P1 Pin43

RX       PA0    analog input     P1 Pin12
RSSI     PA1    analog input     P1 Pin11
TX       PA4    analog output    P1 Pin16

EXT_CLK  PA15   input            P2 Pin40

- Host communication:
USART3 - TXD PC10 - RXD PC11
- Serial repeater (3.3 V):
UART5  - TXD PC12 - RXD PD2
\end{verbatim}

\subsection{Pin definitions for Nucleo-64 F446RE boards (Morpho header):}

\begin{verbatim}
PTT      PB13   output           CN10 Pin30
COSLED   PB14   output           CN10 Pin28
LED      PA5    output           CN10 Pin11
COS      PB15   input            CN10 Pin26

DSTAR    PB10   output           CN10 Pin25
DMR      PB4    output           CN10 Pin27
YSF      PB5    output           CN10 Pin29
P25      PB3    output           CN10 Pin31

MDSTAR   PC4    output           CN10 Pin34
MDMR     PC5    output           CN10 Pin6
MYSF     PC2    output           CN7 Pin35
MP25     PC3    output           CN7 Pin37

RX       PA0    analog input     CN7 Pin28
RSSI     PA1    analog input     CN7 Pin30
TX       PA4    analog output    CN7 Pin32

EXT_CLK  PA15   input            CN7 Pin17

- Host communication:
USART2 - TXD PA2  - RXD PA3  (already connected with ST-Link VCP)
- Serial repeater (3.3 V):
UART5  - TXD PC12 - RXD PD2
\end{verbatim}

\subsection{Pin definitions for Nucleo-64 F446RE boards (Arduino header):}

\begin{verbatim}
PTT      PB10   output           CN9 Pin7
COSLED   PB3    output           CN9 Pin4
LED      PB5    output           CN9 Pin5
COS      PB4    input            CN9 Pin6

DSTAR    PA1    output           CN8 Pin2
DMR      PA4    output           CN8 Pin3
YSF      PB0    output           CN8 Pin4
P25      PC1    output           CN8 Pin5

RX       PA0    analog input     CN8 Pin1
RSSI     PC0    analog input     CN8 Pin6
TX       PA5    analog output    CN5 Pin6

EXT_CLK  PB8    input            CN5 Pin10

- Host communication:
USART2 - TXD PA2  - RXD PA3  (already connected with ST-Link VCP)
- Serial repeater (3.3 V):
USART1 - TXD PA9  - RXD PA10
\end{verbatim}

\subsection{Pin definitions for MMDVM-Pi:}

\begin{verbatim}
PTT      PB13   output
COSLED   PB14   output
LED      PB15   output
COS      PC0    input

DSTAR    PC7    output
DMR      PC8    output
YSF      PA8    output
P25      PC9    output

RX       PA0    analog input
RSSI     PA7    analog input
TX       PA4    analog output

EXT_CLK  PA15   input

- Host communication:
USART1 - TXD PA9  - RXD PA10
- Serial repeater (3.3 V):
UART5  - TXD PC12 - RXD PD2
\end{verbatim}

\subsection{Pin definitions for Nucleo-144 F767ZI boards (Morpho header):}

\begin{verbatim}
PTT      PB13   output           CN12 Pin30
COSLED   PB14   output           CN12 Pin28
LED      PA5    output           CN12 Pin11
COS      PB15   input            CN12 Pin26

DSTAR    PB10   output           CN12 Pin25
DMR      PB4    output           CN12 Pin27
YSF      PB5    output           CN12 Pin29
P25      PB3    output           CN12 Pin31

MDSTAR   PC4    output           CN12 Pin34
MDMR     PC5    output           CN12 Pin6
MYSF     PC2    output           CN11 Pin35
MP25     PC3    output           CN11 Pin37

RX       PA0    analog input     CN11 Pin28
RSSI     PA1    analog input     CN11 Pin30
TX       PA4    analog output    CN11 Pin32

EXT_CLK  PA15   input            CN11 Pin17

- Host communication:
USART3 - TXD PD8  - RXD PD9  (already connected with ST-Link VCP)
- Serial repeater (3.3 V):
UART5  - TXD PC12 - RXD PD2
\end{verbatim}

\section{Final notes}

\begin{itemize}[leftmargin=*]
	
\item You will find more instructions here: \url{https://github.com/juribeparada/MMDVM_man}

\item The source code of this document (LaTeX) is here: \url{https://github.com/juribeparada/MMDVM_man/blob/master/STM32_MMDVM/mmdvm_stm32_build.tex}

\end{itemize}

\end{document}
